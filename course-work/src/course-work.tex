\documentclass[a4paper, 14pt]{extarticle}

%%% Спасибо https://github.com/Amet13/bachelor-diploma/blob/master/inc/preamble.tex
%%% Преамбула %%%

\usepackage{fontspec} % XeTeX
\usepackage{xunicode} % Unicode для XeTeX
\usepackage{xltxtra}  % Верхние и нижние индексы
\usepackage{pdfpages} % Вставка PDF

\usepackage{listings} % Оформление исходного кода
\lstset{
basicstyle=\small\ttfamily, % Размер и тип шрифта
breaklines=true, % Перенос строк
tabsize=2, % Размер табуляции
literate={--}{{-{}-}}2 % Корректно отображать двойной дефис
}

% Шрифты, xelatex
\defaultfontfeatures{Ligatures=TeX}
\setmainfont{Times New Roman} % Нормоконтроллеры хотят именно его
\newfontfamily\cyrillicfont{Times New Roman}
%\setsansfont{Liberation Sans} % Тут я его не использую, но если пригодится
\setmonofont{FreeMono} % Моноширинный шрифт для оформления кода

% Русский язык
\usepackage{polyglossia}
\setdefaultlanguage{russian}

\usepackage{amssymb,amsfonts,amsmath} % Математика
\numberwithin{equation}{section} % Формула вида секция.номер

\usepackage{enumerate} % Тонкая настройка списков
\usepackage{indentfirst} % Красная строка после заголовка
\usepackage{float} % Расширенное управление плавающими объектами
\usepackage{multirow} % Сложные таблицы

% Пути к каталогам с изображениями
\usepackage{graphicx} % Вставка картинок и дополнений

% Формат подрисуночных записей
\usepackage{chngcntr}
\counterwithin{figure}{section}

% Гиперссылки
\usepackage{hyperref}
\hypersetup{
colorlinks, urlcolor={black}, % Все ссылки черного цвета, кликабельные
linkcolor={black}, citecolor={black}, filecolor={black},
}

% Оформление библиографии и подрисуночных записей через точку
\makeatletter
\renewcommand*{\@biblabel}[1]{\hfill#1.}
\renewcommand*\l@section{\@dottedtocline{1}{1em}{1em}}
\renewcommand{\thefigure}{\thesection.\arabic{figure}} % Формат рисунка секция.номер
\renewcommand{\thetable}{\thesection.\arabic{table}} % Формат таблицы секция.номер
\def\redeflsection{\def\l@section{\@dottedtocline{1}{0em}{10em}}}
\makeatother

\renewcommand{\baselinestretch}{1.4} % Полуторный межстрочный интервал
\parindent 1.27cm % Абзацный отступ

\sloppy             % Избавляемся от переполнений
\hyphenpenalty=1000 % Частота переносов
\clubpenalty=10000  % Запрещаем разрыв страницы после первой строки абзаца
\widowpenalty=10000 % Запрещаем разрыв страницы после последней строки абзаца

% Отступы у страниц
\usepackage{geometry}
\geometry{left=3cm}
\geometry{right=1cm}
\geometry{top=2cm}
\geometry{bottom=2cm}

% Списки
\usepackage{enumitem}
\setlist[enumerate,itemize]{leftmargin=12.7mm} % Отступы в списках

\makeatletter
\AddEnumerateCounter{\asbuk}{\@asbuk}{м)}
\makeatother
\setlist{nolistsep} % Нет отступов между пунктами списка
\renewcommand{\labelitemi}{--} % Маркет списка --
\renewcommand{\labelenumi}{\asbuk{enumi})} % Список второго уровня
\renewcommand{\labelenumii}{\arabic{enumii})} % Список третьего уровня

% Содержание
\usepackage{tocloft}
\renewcommand{\cfttoctitlefont}{\hspace{0.38\textwidth}\MakeTextUppercase} % СОДЕРЖАНИЕ
\renewcommand{\cftsecfont}{\hspace{0pt}}            % Имена секций в содержании не жирным шрифтом
\renewcommand\cftsecleader{\cftdotfill{\cftdotsep}} % Точки для секций в содержании
\renewcommand\cftsecpagefont{\mdseries}             % Номера страниц не жирные
\setcounter{tocdepth}{3}                            % Глубина оглавления, до subsubsection

% Нумерация страниц справа сверху
\usepackage{fancyhdr}
\pagestyle{fancy}
\fancyhf{}
\fancyhead[R]{\textrm{\thepage}}
\fancyheadoffset{0mm}
\fancyfootoffset{0mm}
\setlength{\headheight}{17pt}
\renewcommand{\headrulewidth}{0pt}
\renewcommand{\footrulewidth}{0pt}
\fancypagestyle{plain}{
\fancyhf{}
\rhead{\thepage}
}

% Формат подрисуночных надписей
\RequirePackage{caption}
\DeclareCaptionLabelSeparator{defffis}{ -- } % Разделитель
\captionsetup[figure]{justification=centering, labelsep=defffis, format=plain} % Подпись рисунка по центру
\captionsetup[table]{justification=raggedright, labelsep=defffis, format=plain, singlelinecheck=false} % Подпись таблицы слева
\addto\captionsrussian{\renewcommand{\figurename}{Рис.}} % Имя фигуры

% Заголовки секций в оглавлении в верхнем регистре
\usepackage{textcase}
\makeatletter
\let\oldcontentsline\contentsline
\def\contentsline#1#2{
\expandafter\ifx\csname l@#1\endcsname\l@section
\expandafter\@firstoftwo
\else
\expandafter\@secondoftwo
\fi
{\oldcontentsline{#1}{\MakeTextUppercase{#2}}}
{\oldcontentsline{#1}{#2}}
}
\makeatother

% Оформление заголовков
\usepackage[compact,explicit]{titlesec}
\titleformat{\section}{}{}{12.5mm}{\clearpage\centering{\MakeTextUppercase{#1}}\vspace{24pt}}
\titleformat{\subsection}[block]{\vspace{1em}}{}{12.5mm}{\thesubsection\quad#1\vspace{14pt}}
\titleformat{\subsubsection}[block]{\vspace{1em}\normalsize}{}{12.5mm}{\thesubsubsection\quad#1\vspace{1em}}
\titleformat{\paragraph}[block]{\normalsize}{}{12.5mm}{\MakeTextUppercase{#1}}

% Секции без номеров (введение, заключение...), вместо section*{}
\newcommand{\anonsection}[1]{
\clearpage
\phantomsection % Корректный переход по ссылкам в содержании
\paragraph{\centerline{{#1}}\vspace{1.5em}}
\addcontentsline{toc}{section}{\uppercase{#1}}
}

% Секции для приложений
\newcommand{\appsection}[1]{
\phantomsection
\paragraph{\centerline{{#1}}}
\addcontentsline{toc}{section}{\uppercase{#1}}
}

% Библиография: отступы и межстрочный интервал
\makeatletter
\renewenvironment{thebibliography}[1]
{\section*{\refname}
\list{\@biblabel{\@arabic\c@enumiv}}
{\settowidth\labelwidth{\@biblabel{#1}}
\leftmargin\labelsep
\itemindent 16.7mm
\@openbib@code
\usecounter{enumiv}
\let\p@enumiv\@empty
\renewcommand\theenumiv{\@arabic\c@enumiv}
}
\setlength{\itemsep}{0pt}
}
\makeatother

\setcounter{page}{4} % Начало нумерации страниц


\usepackage[figurename=Рис.]{newfloat} %Floating environments
\DeclareGraphicsExtensions{.pdf,.png,.jpg}
\graphicspath{{images/pdf/}{images/png/}{images/jpg/}}


\usepackage[outputdir=\detokenize{../out/}, newfloat=true]{minted}
\setminted[cpp]{
linenos=true,
breaklines=true,
encoding=utf8,
frame=single,
fontsize=\small,
tabsize = 2
}


\usepackage[parentracker=true,
backend=biber,
hyperref=auto,
language=auto,
autolang=other,
citestyle=gost-numeric,
defernumbers=true,
bibstyle=gost-numeric,
]{biblatex}

% Russian sources first
\DeclareSourcemap{
\maps[datatype=bibtex]{
\map{
\step[fieldset=langid, fieldvalue={tempruorder}]
}
\map[overwrite]{
\step[fieldsource=langid, match=russian, final]
\step[fieldsource=presort, match=\regexp{(.+)}, replace=\regexp{aa$1}]  %$
}
\map{
\step[fieldsource=langid, match=russian, final]
\step[fieldset=presort, fieldvalue={az}]
}
\map[overwrite]{
\step[fieldsource=langid, notmatch=russian, final]
\step[fieldsource=presort, match=\regexp{(.+)}, replace=\regexp{za$1}]  %$
}
\map{
\step[fieldsource=langid, notmatch=russian, final]
}
\map{
\step[fieldsource=langid, match={tempruorder}, final]
\step[fieldset=langid, null]
}
}
}
%\xpatchbibmacro{related:default}
%  {\renewbibmacro*{pageref}{}}
%  {\renewbibmacro*{pageref}{}\renewbibmacro*{begentry}{}}{}{}
%\renewbibmacro*{begentry}
%  {\hspace{-4em}\makebox[4em]{\hyperlink{\thefield{entrykey}}{$\Downarrow$}}%
%       \raisebox{\baselineskip}{\hypertarget{back:\thefield{entrykey}}{}}}%
\edef\mytempforAtCharacter{\char64}
\def\getkey#1#2{%
\csappto{mytempforbibkey}{#1}%
\ifstrequal{#2}{,}
{\hspace{-4em}\makebox[4em]{\hyperlink{back:\mytempforbibkey}{$\Uparrow$}}%
\raisebox{\baselineskip}{\hypertarget{\mytempforbibkey}{}}%
\mytempforAtCharacter\mytempforbibtype\{\mytempforbibkey,}
{\getkey#2}%
}
\renewcommand*{\do}[1]{\csdef{#1}##1{\def\mytempforbibtype{#1}\def\mytempforbibkey{}\getkey}}
\docsvlist{MVBOOK,BOOK,INBOOK,SUPPBOOK,BOOKINBOOK,MVCOLLECTION,COLLECTION,INCOLLECTION,SUPPCOLLECTION,MVREFERENCE,REFERENCE,INREFERENCE,MVPROCEEDINGS,PROCEEDINGS,INPROCEEDINGS,PERIODICAL,ARTICLE,PATENT,ONLINE,THESIS,mvbook,book,inbook,suppbook,bookinbook,mvcollection,collection,incollection,suppcollection,mvreference,reference,inreference,mvproceedings,proceedings,inproceedings,periodical,article,patent,online,thesis,Article,Book,MVBook,InBook,BookInBook,SuppBook,Booklet,Collection,MVCollection,InCollection,SuppCollection,Manual,Misc,Online,Patent,Periodical,SuppPeriodical,Proceedings,MVProceedings,InProceedings,Reference,MVReference,InReference,Report,Thesis,Unpublished}
\renewcommand*{\do}[1]{\csdef{#1}##1{\mytempforAtCharacter #1\{}}
\docsvlist{XDATA,COMMENT,xdata,comment,}

\setcounter{biburllcpenalty}{7000}
\setcounter{biburlucpenalty}{8000}
\addbibresource{sources.bib}

\begin{document}

	\includepdf{title-page}

	\tableofcontents

	\begin{anonsection}{Введение}
		В данной работе будет рассмотрен процесс создания простой игры.
		В качестве эталона взята логика игры <<Atomic Bomber>>, реализация будет выполнена на языке C++ на основе легковесного игрового движка Urho3D.
		В ходе выполнения работы будут выполнены следующие задачи:
		\begin{enumerate}
			\item декомпозиция задачи;
			\item реализация основных компонентов игры с использованием комбинации подходов объектно-ориентированного, компонентно-ориентированного и событийно-ориентированного программирования;
			\item тестирование и отладка приложения.
		\end{enumerate}

		В результате должны быть закреплены и расширены навыки программирования с использованием выбранного набора программных средств и получен работающий прототип игры.

	\end{anonsection}

	\begin{section}{Обзор предметной области}

		\begin{subsection}{Постановка задачи}

			В качестве темы данной работы была выбрана реализация игры, повторяющей основные аспекты геймплея игры <<Atomic Bomber>> за авторством Luke Allen.
			Это двухмерная игра для мобильных устройств, работающая по простыми правилам, которые, тем не менее, формируют интересный игровой процесс.

			<<Atomic Bomber>> была выбрана в качестве эталона реализации игровой логики в данной работе, так как игровая логика максимально проста и может быть реализована за разумное время, но для ее корректной реализации потребуется применить различные техники программирования, которые будут рассмотрены далее.

			Приведем основные аспекты игровой логики, которые будут реализованы.

			\begin{enumerate}
				\item игровой мир двухмерный, вид сбоку;
				\item нижнюю часть экрана занимает случайно сгенерированный ландшафт, на поверхности которого могут находиться противники;
				\item игрок управляет самолетом, имеющим постоянную скорость, правая и левые стороны "закольцованы" друг на друга (т.е. пролет через них телепортирует игрока к противоположной границе), вылет за верхнюю границу также невозможен.
				\item самолет вооружен бомбами, деформирующими ландшафт при столкновении и уничтожающими находящихся в определенном радиусе противников;
				\item при столкновении самолета с ландшафтом уровень перезагружается.
			\end{enumerate}

			Скриншот оригинальной игры представлен на рисунке~\ref{atomic-bomber}.

			\begin{figure}[H]
				\includegraphics[width=\linewidth]{atomic-bomber}
				\caption{Скриншот игрового процесса оригинального <<Atomic Bomber>>}
				\label{atomic-bomber}
			\end{figure}
		\end{subsection}

		\begin{subsection}{Обзор инструментальных средств}
			Программа будет разработана на языке C++.
			Так как самостоятельная реализация даже самого базового графического фреймворка для разработки нашего приложения очень трудоемка, будем использовать движок Urho3D в качестве такого фреймворка.
			Он позволяет абстрагироваться от низкоуровневых деталей разработки игры, предоставляя реализации таких необходимых модулей как:
			\begin{enumerate}
				\item управление жизненным циклом программы, обработка событий ОС;
				\item компонентно-ориентированная модель для описания объектов в 2D/3D среде (сцене), легковесный визуальный редактор сцен;
				\item загрузка текстур и других ассетов в различных форматах, рендеринг 2D и 3D графики, вывод звука, рассчет физики, обработка пользовательского ввода и другие важные подисистемы.
			\end{enumerate}

			Рассмотрим подробнее три парадигмы программирования, которые мы будем использовать:

			Объектно-ориентированное программирование - наиболее популярная парадигма программирования на данный момент, поддерживается C++ на языковом уровне.
			Данный подход позволяет моделировать предметную область как набор объектов, взаимодействующих друг с другом.

			Компонентно-ориентированное программирование - широко используемый при разработки игр подход, максимально использущий композицию вместо наследования (<<composition over inheritance>>).
			Существуют различные его вариации, но в данной работе мы будем использовать модель, предоставляемую движком Urho3D, в котором данный подход используется для описания структуры игровой сцены.
			Компонентно-ориентированный подход к построению сцены позволяет удобно реализовывать логику взаимодействия игровых объектов в 2D/3D пространстве и инкапсулировать отдельные ее аспекты.
			Объекты на сцене (называемые \verb|Node|) являются C++ классами, не предоставляющими сами по себе никакой функционал кроме их идентификации и координат на сцене, но при этом являющимися контейнерами компонентов.
			С помощью таких методов, как \verb|GetComponent<TComponent>| и \verb|AddComponent<TComponent>()|, программист может описывать функционал объекта набором ассоциированных с ним компонентов.
			Например, игровой ландшафт может быть представлен объектом с компонентами логической модели ландшафта, его физической модели и компонента отрисовки в спрайт.
			Сами компоненты также являются C++ классами, унаследованными от \verb|LogicComponent|.

			Событийно-ориентированное программирование - используемый в различных областях подход, описывающий систему как набор событий и действий, ассоциированных с активацией этих событий.
			В нашей программе мы будем использовать реализацию событий с помощью шаблона проектирования <<Наблюдатель>> (<<Observer>>) на основе указателей на функции C++ для коммуникации между компонентами объектов.
			Для этого воспользуемся открытой библиотекой Signals~\cite{signals}, предоставляющей функционал сигналов и слотов, похожий на делегаты и события в C\#.

			Для поддержания качества кода использовались диагностики IDE CLion и статический анализатор кода PVS-Studio.

			При разработке использовались порталы документация языка C++~\cite{cppreference}, API Urho3D~\cite{urho3d-docs}, а также книга Теплякова С~\cite{teplyakov}, описывающая распространенные шаблоны проектирования.
		\end{subsection}


	\end{section}

	\begin{section}{Программная реализация игры}

		\begin{subsection}{Инициализация}
			Входной точкой нашего приложения является класс GameApplication, который регистрируется в движке Urho3D макросом \verb|Urho3D_DEFINE_APPLICATION_MAIN(GameApplication)|.
			В нем производится получение контекста от Urho3D, начальная настройка приложения, вызов регистрации классов компонентов, а также создание и регистрация подсистемы управления игровым состоянием GameSubsystem (в Urho3D подсистемами называются объекты, существующие в единственном экземпляре во всем приложении).

			GameSubsystem предоставляет функционал загрузки, выгрузки и перезагрузки нашего игрового уровня.
			Компоненты на нем, в свою очередь, реагируют на различные события (например, переход на следующий кадр, пользовательский ввод или столкновения), формируя игровой процесс.
		\end{subsection}

		\begin{subsection}{Структура уровня}
			Уровень (сцена) представляет собой иерархию объектов (называемых Node) и их компонентов:

			\includegraphics[width=250]{scene}

			\begin{enumerate}
				\item объект MainCamera в нашем случае является закрепленной на одном месте ортогональной (не перспективной, т.к. наш мир двухмерен) камерой, автоматически масштабируется при запуске для корректного отображения игровой области;
				\item объект Terrain представляет ландшафт;
				\item объект Aircraft представляет самолет игрока;
				\item объект Sky отображает спрайт неба на заднем плане;
				\item объекты LeftBound, RightBound и UpperBound расположены по краям видимой области и содержат физические коллайдеры, не позволяющие игроку покинуть пределы видимого пространства.
			\end{enumerate}

			Рассмотрим их подробнее.
		\end{subsection}

		\begin{subsection}{Ландшафт}
			Наиболее комплексная система в нашем приложении, состоит из трех компонентов
			\begin{enumerate}
				\item \verb|TerrainController| предоставляет логическую модель ландшафта, с которой работают другие компоненты;
				\item \verb|TerrainSpriteController| занимается динамической отрисовкой спрайта ландшафта;
				\item \verb|TerrainCollisionShapeController| выполняет динамическое построение физической модели ландшафта для работы столкновений.
			\end{enumerate}

			\verb|TerrainController| содержит карту высот: одномерный массив чисел с плавающей точкой, содержащий высоты в соответствующих точках ландшафта.
			Предоставляет методы для деформации ландшафта (взрыва), перевода координат из относительных координат (например, 0.5 будет серединой ландшафта) в абсолютные (мировые), а также событие об изменени ландшафта, позволяющее подписчикам отреагировать на это изменение.

			В начале игры карта высот генерируется одномерной версией алгоритма MidpointDisplacement, который рекурсивно генерирует случайную высоту в середине заданного промежутка на основании высот краев:
			\inputminted{cpp}{listings/MidpointDisplacement1D.cpp}

			При любых изменениях карты высот вызывается событие \verb|HeightmapUpdated|, позволяющее указанным выше компонентам перерисовать спрайт и перестроить сетку коллизий.

			Пример генерируемого ландшафта приведен на рисунке~\ref{terrain}.

			\begin{figure}[h]
				\includegraphics[width=\linewidth]{terrain}
				\caption{Пример сгенерированного ландшафта}
				\label{terrain}
			\end{figure}
		\end{subsection}

		\begin{subsection}{Игрок}
			Поведение самолета игрока контроллируют следующие компоненты:
			\begin{enumerate}
				\item \verb|AircraftController| подписывается на событие столкновения с ландшафтом (для вызова \verb|GameSubsystem::RequestReloadGameLevel| при столкновении), создает остальные компоненты;
				\item \verb|AircraftMovingController| контроллирует движение самолета: поддерживает скорость и направление движения, обрабатывает коллизиии с краями мира;
				\item \verb|AircraftMouseController| обрабатывает пользовательские клики мышью и отправляет соответствующие команды на изменение направления движения самолета (полет к заданной точке);
				\item \verb|AircraftBombsController| обрабатывает нажатие клавиши сброса бомб, инстанцирует объект бомбы.
			\end{enumerate}
		\end{subsection}

		\begin{subsection}{Коллизии}
			Обработка коллизий ведется компонентом \verb|CollisionsAggregator|, который должен быть присутствовать у всех динамических объектов, заинтересованных в регистрации столкновений со статическим окружением.
			Оборачивает внутренние физические события Urho3D и предоставляет события, соответствующие каждому из типов коллайдеров (описаны далее).
			Например, так выглядит подписка самолета на свое столкновение с ландшафтом:

			\inputminted{cpp}{listings/AircraftCollisionExample.cpp}

			Для того, чтобы была возможность раздельно обрабатывать столкновения с тремя коллайдерами краев мира и ландшафтом, был введен базовый компонент \verb|StaticColliderComponent|, каждый из наследников которого идентифицирует свой тип коллайдера.

			Благодаря такому подходу, \verb|CollisionsAggregator| при регистрации столкновения просто проверяет наличие \verb|StaticColliderComponent| на столкнувшемся объекте и наличие идентификатора его конкретного типа в своем внутреннем словаре соответствующих событий.
			Если для этого типа коллайдера было создано событие во внутреннем словаре (\verb|HashMap|), то оно будет вызвано.

			Выделение этого функционала в универсальный компонент \verb|CollisionsAggregator| позволило избежать дублирования однотипного кода подписок на столкновения в компонентах самолета, бомб и т.д.
		\end{subsection}

		\begin{subsection}{Снаряды}
			При создании объекта бомбы на ней инициализируется компонент \verb|ShellController|, который в свою очередь обрабатывает движение и столкновения с границами мира, а также инициализирует компонент \verb|ExplosivesController|.
			Последний содержит характеристики взрыва и статическое событие \verb|SomethingExploded|, позволяющее таким объектам как ландшафт и противникам отреагировать на взрыв соответствующим образом.
			В частности, ландшафт деформируется, а противник будет уничтожен, если находится в радиусе.

			Такая событийно-ориентированная организация позволяет в дальнейшем прикрепить этот компонент не только к бомбам, но и к любым другим снарядам, в том числе и направленным против игрока.
			Потребуется лишь указать характеристики, а игроку подписаться на событие взрыва.
		\end{subsection}

		\begin{subsection}{Противники}
			На данный момент представлены только танками.
			Структура компонентов практически не отличается от самолета, но вместо контроля игроком выполняется простое движение влево/вправо с заданной скоростью с разворотом у границ мира.
		\end{subsection}

		\begin{subsection}{Тестирование}
			Проводились тесты на одновременный выброс большого количества бомб, в том числе и в пролете через границу мира.
			Также тестировалась многоразовая перезагрузка уровня в разных условиях.

			В процессе реализации логики ландшафта была обнаружена и исправлена утечка памяти в используемом внутри Urho3D физическом движке \verb|Box2D| (\url{https://github.com/Urho3D/Urho3D/issues/1995}).
			Обнаружить и локализовать проблему помог динамический анализатор кода \verb|Valgrind|.

			Также регулярно проводились проверки статическим анализатором кода PVS-Studio.
		\end{subsection}

	\end{section}

	\begin{anonsection}{Заключение}
		В данной курсовой работе удалось реализовать запланированный функционал программы с использованием средства языка C++, а также различных библиотек и инструментов.

		Была использована на практике комбинация объектно-ориентированного, компонентно-ориентированного и событийно-ориентированного подходов программирования в совокупности с различными технологиями и инструментами разработки программного обеспечения.

		Исходный код проекта и текст данной работы доступны по адресу \url{https://github.com/ArXen42/AlienBomberUrho3D}.
	\end{anonsection}

	\printbibliography[title = Список использованных источников]

\end{document}
