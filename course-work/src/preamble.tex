\documentclass[a4paper, 14pt]{extarticle}

%%% Спасибо https://github.com/Amet13/bachelor-diploma/blob/master/inc/preamble.tex
%%% Преамбула %%%

\usepackage{fontspec} % XeTeX
\usepackage{xunicode} % Unicode для XeTeX
\usepackage{xltxtra}  % Верхние и нижние индексы
\usepackage{pdfpages} % Вставка PDF

\usepackage{listings} % Оформление исходного кода
\lstset{
basicstyle=\small\ttfamily, % Размер и тип шрифта
breaklines=true, % Перенос строк
tabsize=2, % Размер табуляции
literate={--}{{-{}-}}2 % Корректно отображать двойной дефис
}

% Шрифты, xelatex
\defaultfontfeatures{Ligatures=TeX}
\setmainfont{Times New Roman} % Нормоконтроллеры хотят именно его
\newfontfamily\cyrillicfont{Times New Roman}
%\setsansfont{Liberation Sans} % Тут я его не использую, но если пригодится
\setmonofont{FreeMono} % Моноширинный шрифт для оформления кода

% Русский язык
\usepackage{polyglossia}
\setdefaultlanguage{russian}

\usepackage{amssymb,amsfonts,amsmath} % Математика
\numberwithin{equation}{section} % Формула вида секция.номер

\usepackage{enumerate} % Тонкая настройка списков
\usepackage{indentfirst} % Красная строка после заголовка
\usepackage{float} % Расширенное управление плавающими объектами
\usepackage{multirow} % Сложные таблицы

% Пути к каталогам с изображениями
\usepackage{graphicx} % Вставка картинок и дополнений

% Формат подрисуночных записей
\usepackage{chngcntr}
\counterwithin{figure}{section}

% Гиперссылки
\usepackage{hyperref}
\hypersetup{
colorlinks, urlcolor={black}, % Все ссылки черного цвета, кликабельные
linkcolor={black}, citecolor={black}, filecolor={black},
}

% Оформление библиографии и подрисуночных записей через точку
\makeatletter
\renewcommand*{\@biblabel}[1]{\hfill#1.}
\renewcommand*\l@section{\@dottedtocline{1}{1em}{1em}}
\renewcommand{\thefigure}{\thesection.\arabic{figure}} % Формат рисунка секция.номер
\renewcommand{\thetable}{\thesection.\arabic{table}} % Формат таблицы секция.номер
\def\redeflsection{\def\l@section{\@dottedtocline{1}{0em}{10em}}}
\makeatother

\renewcommand{\baselinestretch}{1.4} % Полуторный межстрочный интервал
\parindent 1.27cm % Абзацный отступ

\sloppy             % Избавляемся от переполнений
\hyphenpenalty=1000 % Частота переносов
\clubpenalty=10000  % Запрещаем разрыв страницы после первой строки абзаца
\widowpenalty=10000 % Запрещаем разрыв страницы после последней строки абзаца

% Отступы у страниц
\usepackage{geometry}
\geometry{left=2cm}
\geometry{right=1cm}
\geometry{top=2cm}
\geometry{bottom=2cm}

% Списки
\usepackage{enumitem}
\setlist[enumerate,itemize]{leftmargin=1.82cm} % Отступы в списках

\makeatletter
\AddEnumerateCounter{\asbuk}{\@asbuk}{м)}
\makeatother
\setlist{nolistsep} % Нет отступов между пунктами списка
\renewcommand{\labelitemi}{--} % Маркет списка --
\renewcommand{\labelenumi}{\asbuk{enumi})} % Список второго уровня
\renewcommand{\labelenumii}{\arabic{enumii})} % Список третьего уровня

% Содержание
\usepackage{tocloft}
\renewcommand\cftsecafterpnum{\vskip0pt}
\renewcommand\cftsubsecafterpnum{\vskip0pt}
\setlength\cftbeforesecskip{0pt}
\setlength\cftbeforesubsecskip{0pt}

\renewcommand{\cfttoctitlefont}{\hspace{0.38\textwidth}\MakeTextUppercase} % СОДЕРЖАНИЕ
\renewcommand{\cftsecfont}{\hspace{0pt}}            % Имена секций в содержании не жирным шрифтом
\renewcommand\cftsecleader{\cftdotfill{\cftdotsep}} % Точки для секций в содержании
\renewcommand\cftsecpagefont{\mdseries}             % Номера страниц не жирные
\setcounter{tocdepth}{3}                            % Глубина оглавления, до subsubsection


%% Нумерация страниц справа сверху
%\usepackage{fancyhdr}
%\pagestyle{fancy}
%\fancyhf{}
%\fancyhead[R]{\textrm{\thepage}}
%\fancyheadoffset{0mm}
%\fancyfootoffset{0mm}
%\setlength{\headheight}{17pt}
%\renewcommand{\headrulewidth}{0pt}
%\renewcommand{\footrulewidth}{0pt}
%\fancypagestyle{plain}{
%\fancyhf{}
%\rhead{\thepage}
%}

% Формат подрисуночных надписей
\RequirePackage{caption}
\DeclareCaptionLabelSeparator{defffis}{ -- } % Разделитель
\captionsetup[figure]{justification=centering, labelsep=defffis, format=plain} % Подпись рисунка по центру
\captionsetup[table]{justification=raggedright, labelsep=defffis, format=plain, singlelinecheck=false} % Подпись таблицы слева

% Заголовки секций в оглавлении в верхнем регистре
\usepackage{textcase}
\makeatletter
\let\oldcontentsline\contentsline
\def\contentsline#1#2{
\expandafter\ifx\csname l@#1\endcsname\l@section
\expandafter\@firstoftwo
\else
\expandafter\@secondoftwo
\fi
{\oldcontentsline{#1}{\MakeTextUppercase{#2}}}
{\oldcontentsline{#1}{#2}}
}
\makeatother

% Оформление заголовков
\usepackage[compact,explicit]{titlesec}
\titleformat{\section}{}{}{12.5mm}{\clearpage\centering{\MakeTextUppercase{#1}}\vspace{24pt}}
\titleformat{\subsection}[block]{\vspace{0pt}}{}{12.5mm}{\thesubsection\quad\textbf{#1}\vspace{24pt}}
\titleformat{\subsubsection}[block]{\vspace{1em}\normalsize}{}{12.5mm}{\thesubsubsection\quad#1\vspace{1em}}
\titleformat{\paragraph}[block]{\normalsize}{}{12.5mm}{\MakeTextUppercase{#1}}

\titlespacing*{\section}
{0pt}{5.5ex plus 1ex minus .2ex}{4.3ex plus .2ex}
\titlespacing*{\subsection}
{0pt}{5.5ex plus 1ex minus .2ex}{4.3ex plus .2ex}

% Секции без номеров (введение, заключение...), вместо section*{}
\newcommand{\anonsection}[1]{
\clearpage
\phantomsection % Корректный переход по ссылкам в содержании
\paragraph{\centerline{{#1}}\vspace{1.5em}}
\addcontentsline{toc}{section}{\uppercase{#1}}
}

% Секции для приложений
\newcommand{\appsection}[1]{
\phantomsection
\paragraph{\centerline{{#1}}}
\addcontentsline{toc}{section}{\uppercase{#1}}
}

% Библиография: отступы и межстрочный интервал
\makeatletter
\renewenvironment{thebibliography}[1]
{\section*{\refname}
\list{\@biblabel{\@arabic\c@enumiv}}
{\settowidth\labelwidth{\@biblabel{#1}}
\leftmargin\labelsep
\itemindent 16.7mm
\@openbib@code
\usecounter{enumiv}
\let\p@enumiv\@empty
\renewcommand\theenumiv{\@arabic\c@enumiv}
}
\setlength{\itemsep}{0pt}
}
\makeatother

\usepackage[figurename=Рисунок]{newfloat} %Floating environments
\DeclareGraphicsExtensions{.pdf,.png,.jpg}
\graphicspath{{images/pdf/}{images/png/}{images/jpg/}}


\usepackage[outputdir=\detokenize{../out/}, newfloat=true]{minted}
\setminted[cpp]{
linenos=true,
breaklines=true,
encoding=utf8,
frame=single,
fontsize=\small,
tabsize = 2
}


\usepackage[parentracker=true,
backend=biber,
hyperref=auto,
language=auto,
autolang=other,
citestyle=gost-numeric,
defernumbers=true,
bibstyle=gost-numeric,
]{biblatex}

% Russian sources first
\DeclareSourcemap{
\maps[datatype=bibtex]{
\map{
\step[fieldset=langid, fieldvalue={tempruorder}]
}
\map[overwrite]{
\step[fieldsource=langid, match=russian, final]
\step[fieldsource=presort, match=\regexp{(.+)}, replace=\regexp{aa$1}]  %$
}
\map{
\step[fieldsource=langid, match=russian, final]
\step[fieldset=presort, fieldvalue={az}]
}
\map[overwrite]{
\step[fieldsource=langid, notmatch=russian, final]
\step[fieldsource=presort, match=\regexp{(.+)}, replace=\regexp{za$1}]  %$
}
\map{
\step[fieldsource=langid, notmatch=russian, final]
}
\map{
\step[fieldsource=langid, match={tempruorder}, final]
\step[fieldset=langid, null]
}
}
}
%\xpatchbibmacro{related:default}
%  {\renewbibmacro*{pageref}{}}
%  {\renewbibmacro*{pageref}{}\renewbibmacro*{begentry}{}}{}{}
%\renewbibmacro*{begentry}
%  {\hspace{-4em}\makebox[4em]{\hyperlink{\thefield{entrykey}}{$\Downarrow$}}%
%       \raisebox{\baselineskip}{\hypertarget{back:\thefield{entrykey}}{}}}%
\edef\mytempforAtCharacter{\char64}
\def\getkey#1#2{%
\csappto{mytempforbibkey}{#1}%
\ifstrequal{#2}{,}
{\hspace{-4em}\makebox[4em]{\hyperlink{back:\mytempforbibkey}{$\Uparrow$}}%
\raisebox{\baselineskip}{\hypertarget{\mytempforbibkey}{}}%
\mytempforAtCharacter\mytempforbibtype\{\mytempforbibkey,}
{\getkey#2}%
}
\renewcommand*{\do}[1]{\csdef{#1}##1{\def\mytempforbibtype{#1}\def\mytempforbibkey{}\getkey}}
\docsvlist{MVBOOK,BOOK,INBOOK,SUPPBOOK,BOOKINBOOK,MVCOLLECTION,COLLECTION,INCOLLECTION,SUPPCOLLECTION,MVREFERENCE,REFERENCE,INREFERENCE,MVPROCEEDINGS,PROCEEDINGS,INPROCEEDINGS,PERIODICAL,ARTICLE,PATENT,ONLINE,THESIS,mvbook,book,inbook,suppbook,bookinbook,mvcollection,collection,incollection,suppcollection,mvreference,reference,inreference,mvproceedings,proceedings,inproceedings,periodical,article,patent,online,thesis,Article,Book,MVBook,InBook,BookInBook,SuppBook,Booklet,Collection,MVCollection,InCollection,SuppCollection,Manual,Misc,Online,Patent,Periodical,SuppPeriodical,Proceedings,MVProceedings,InProceedings,Reference,MVReference,InReference,Report,Thesis,Unpublished}
\renewcommand*{\do}[1]{\csdef{#1}##1{\mytempforAtCharacter #1\{}}
\docsvlist{XDATA,COMMENT,xdata,comment,}

\setcounter{biburllcpenalty}{7000}
\setcounter{biburlucpenalty}{8000}
\addbibresource{sources.bib}